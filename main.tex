\documentclass[UTF8]{ctexart}
\usepackage{geometry}
\usepackage{fancyhdr}
\usepackage{graphicx}
\usepackage{array}
\newcommand{\PreserveBackslash}[1]{\let\temp=\\#1\let\\=\temp}
\newcolumntype{C}[1]{>{\PreserveBackslash\centering}p{#1}}
\newcolumntype{R}[1]{>{\PreserveBackslash\raggedleft}p{#1}}
\newcolumntype{L}[1]{>{\PreserveBackslash\raggedright}p{#1}}
% \author{ARZhu}
\author{\zihao{-3}第X试}
\title{\zihao{2}XXX信息竞赛(XXOI201X)}
\date{}
\geometry{left=3.18cm,right=3.18cm,top=2.54cm,bottom=2.54cm}
\begin{document}

\maketitle
\thispagestyle{empty}
\begin{center}
\zihao{-3}\textbf{(请选手务必仔细阅读本页内容)}
\end{center}

\textbf{一、题目概况}
\begin{center}
\begin{tabular}{*{4}{|C{9em}}|}
\hline
    中文题目名称 & & & \\ \hline
    英文题目与子目录名 &  &  &  \\ \hline
    可执行文件名 &  &  & \\ \hline
    输入文件名 & .in & .in & .in \\ \hline
    输出文件名 & .out & .out & .out \\ \hline
    每个测试点时限 & 1秒 & 1秒 & 1秒 \\ \hline
    内存上限 & 128M & 128M & 128M \\ \hline
    测试点数目 & 10 & 10 & 10 \\ \hline
    每个测试点分值 & 10 & 10 & 10 \\ \hline
    附加样例文件 & 有 & 有 & 有 \\ \hline
    结果比较方式 & \multicolumn{3}{|c|}{全文比较(过滤行末空格及文末回车)} \\ \hline
    题目类型 & 传统 & 传统 & 传统 \\ \hline

\end{tabular}
\end{center}

\textbf{二、提交源程序程序名}
\begin{center}
\begin{tabular}{*{4}{|C{9em}}|}
\hline
    对于C++语言 & .cpp & .cpp & .cpp \\ \hline
    对于C语言 & .c & genes.c & .c \\ \hline
    对于pascal语言 & .pas & .pas & .pas \\ \hline
\end{tabular}
\end{center}

%\textbf{三、优化开关}
%\begin{center}
%\begin{tabular}{*{4}{|C{9em}}|}
%\hline
%    对于C++语言 & -O2\ -lm	& -O2\ -lm & -O2\ -lm \\ \hline
%    对于C语言 & -O2\ -lm & -O2\ -lm & -O2\ -lm \\ \hline
%    对于pascal语言 & -O2 & -O2 & -O2 \\ \hline
%\end{tabular}
%\end{center}

\textbf{注意事项:}
\begin{enumerate}
    \item{文件名(程序名和输入输出文件名)必须使用英文小写。}
    \item{C/C++中函数main()的返回类型必须是int,程序正常结束时的返回值必须是0。}
    \item{评测时采用的机器配置为:CPU P4.30GHz,内存1G,上述时限以此配置为准。}
    \item{特别提醒:评测在NOI Linux下进行。}
\end{enumerate}

\newpage
\setcounter{page}{1}
\pagestyle{plain}
\pagenumbering{arabic}

% T1
\newpage
\section{}
\begin{center}
\tt\large{(.cpp/c/pas)}
\end{center}
\subsection{问题描述}

\subsection{输入}
\subsection{输出}

\subsection{输入输出样例1}
\subsubsection{输入样例}
\subsubsection{输出样例}

\subsection{输入输出样例2}
\subsubsection{输入样例}
\subsubsection{输出样例}


\subsection{约定和数据范围}

% T2
\newpage
\section{}
\begin{center}
\tt\large{(.cpp/c/pas)}
\end{center}
\subsection{问题描述}

\subsection{输入}
\subsection{输出}

\subsection{输入输出样例1}
\subsubsection{输入样例}
\subsubsection{输出样例}

\subsection{输入输出样例2}
\subsubsection{输入样例}
\subsubsection{输出样例}


\subsection{约定和数据范围}

% T3
\newpage
\section{}
\begin{center}
\tt\large{(.cpp/c/pas)}
\end{center}
\subsection{问题描述}

\subsection{输入}
\subsection{输出}

\subsection{输入输出样例1}
\subsubsection{输入样例}
\subsubsection{输出样例}

\subsection{输入输出样例2}
\subsubsection{输入样例}
\subsubsection{输出样例}


\subsection{约定和数据范围}


\end{document}
